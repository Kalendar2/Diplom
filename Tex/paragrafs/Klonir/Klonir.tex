\quad 1.\quad В ходе работы с виртуальной машиной возникла необходимость перенести её с одного своего устройства на другое.
\newline Для этого открываем Virtual Box, нажимаем правой кнопкой мыши по своей виртуальной машине и выбираем пункт клонировать. В появившемся окне указываем имя нового клона и его путь по которому он будет сохранён. Так же в графе «Политика MAC-адреса» выбираем вариант: «Сгенерировать новые MAC-адреса всех сетевых адаптеров».

\begin{figure}[h]		
		\centering
		\includegraphics[width=0.5\linewidth]{VM/8.png}
\caption{Окно клонирования.}
\label{ris:image}
\end{figure}

\quad В следующем окне нужно было указать тип клонирования: полное или связное. При связном клонировании будет создана новая машина, использующая файлы виртуальных жёстких дисков клонируемой машины и нельзя перенести её на другой компьютер без переноса клонируемой. При полном клонировании, будет создана полная копия клонируемой виртуальной машины (включая все файлы виртуальных жёстких дисков). Поэтому выбиираем полное клонирование.

\quad В окне с указанием цели клонирования, указываем клонировать всё, чтобы новая машина не только отражала текущее состояние клонируемой машины, но и имела копии всех снимков её древа снимков.

\quad Далее нажимаем на кнопку «клонировать», после чего запускается процесс клонирования. По его завершению переносим новую машину на флэшку. Это можно сделать нажав в Virtual Box на клон правой кнопкой мыши и выбрать пункт «Переместить». Или просто зайти в папку, в которую был сохранён наш клон, и переместить его уже оттуда.

\quad Далее подсоединяем флэшку к другому компьютеру и переносим машину в папку Virtual Box. Открыв Virtual Box, нажиимаем вверху на кнопку «Машина» и выбрал пункт добавить. После чего находим свою машину и нажал кнопку «Открыть».

\begin{figure}[h]
		\centering
		\includegraphics[width=0.4\linewidth]{VM/9.png}
\caption{Добавление виртуальной машины.}
\label{ris:image}
\end{figure}

\quad Виртуальная машина добавлена. Но зайдя в настройки, можно увидеть, что объём выделенной основной памяти составляет всего лишь 2 гигабайта, что слишком мало для работы с машиной. Так как наша машина находится в состоянии «Сохранена», мы не можем изменять её настройки. Поэтому нажимаем правой кнопкой мыши по перенесённому клону, и выбираем пункт «Сбросить сохранённое состояние». После чего снова нажимаем правой кнопкой мыши по машине, выбираем пункт «Настроить…» и в «Системе» выделяем нужное количество памяти.

\begin{figure}[h]
		\centering
		\includegraphics[width=0.75\linewidth]{VM/10.png}
\caption{Настройка памяти.}
\label{ris:image}
\end{figure}

\quad Далее заходим в «Носители» и выбираем свой жёсткий диск, так как иначе при запуске виртуальной машины мы бы ничего не увидели.

\begin{figure}[h]
		\centering
		\includegraphics[width=0.75\linewidth]{VM/11.png}
\caption{Настройки. Носители.}
\label{ris:image}
\end{figure}

\quad Теперь клон виртуальной машины перемещён, добавлен на новый компьютер и с ним можно работать.

\quad Есть и альтернативный способ переноса виртуальной машины с одного устройства на другое, с помощью функций «Экспорт» и «Импорт».

\begin{figure}[h]
\centering Экспорт виртуальной машины
\label{ris:image}
\end{figure}

\quad Экспорт конфигурации виртуальной машины происходит в файл формата .ova (Open Virtual Appliance). Это универсальный формат для хранения данных виртуальной машины, файлы .ova могут использоваться в разных программах виртуализации: VirtualBox, VMware Workstation, Microsoft Hyper-V. Виртуальная машина, экспортированная в файл .ova, затем может быть импортирована как в VirtualBox, так и в VMware Workstation, Microsoft Hyper-V.

\quad В меню программы нужно зайти в «Файл» и выбрать пункт «Экспорт конфигураций». В открывшемся окне выбираем машину для экспорта, и нажимаем «Далее».

\begin{figure}[h]
		\centering
		\includegraphics[width=1\linewidth]{VM/12.png}
\caption{«Экспорт конфигураций».}
\label{ris:image}

\end{figure}

\quad Выбираем место размещения после экспорта. Также лучше выбрать «Включать МАС-адреса всех сетевых адаптеров», нажимаем «Далее». 

\begin{figure}[h]
		\centering
		\includegraphics[width=1\linewidth]{VM/13.png}
\caption{МАС-адреса сетевых адаптеров.}
\label{ris:image}
\end{figure}

\quad В следующем окне оставляем всё без изменений, и нажимаем кнопку “Экспорт”. Сам экспорт может занимать несколько минут, в зависимости от размера виртуальной машины. После экспорта в указанном месте создается файл, который уже необходимо будет импортировать.

\begin{figure}[h]
\centering Импорт виртуальной машины
\label{ris:image}
\end{figure}

\quad Теперь необходимо скопировать файл на флэшку. Или же можно воспользоваться облочным хранилищем, так как возможно при попытке перенести его на флешку, может возникнуть ошибка: «Файл слишком велик для конечной файловой системы». Она возникает, если передаётся файл размером более четырёх гигабайт на носитель, неспособный с ним работать. Для устранения этой ошибки можно воспользоваться форматированием флешки или разбитием файла с виртуальной машины на несколько частей. Был также опробован способ сжатия виртуальной машины в zip файл, но она оказалась практически без сжимающий составляющих.

\quad Далее на втором компьютере заходим в программу Virtual Box и нажимаем вверху «Файл» и выбираем пункт «Импорт конфигураций».

\begin{figure}[h]
		\centering
		\includegraphics[width=0.85\linewidth]{VM/14.png}
\caption{Импорт конфигураций.}
\label{ris:image}
\end{figure}

\quad В окне импорта выбираем место размещения файла виртуальной машины, нажимаем «Далее». В следующем окне можно изменить параметры импорта, например, увеличить количество процессоров. Также желательно «Включать (сгенерировать новые) МАС-адреса всех сетевых адаптеров», и нажимаем «Импорт».

\quad Импорт также как и экспорт в зависимости от размера виртуальной машины может занимать несколько минут.

\quad После импорта виртуальная машина появляется в списке и с ней уже можно будет работать.