\documentclass[oneside,final,12pt]{extarticle} %размер бумаги устанавливаем А4, шрифт 12пунктов
\usepackage[utf8]{inputenc}
\usepackage[T2A]{fontenc}
\usepackage[english,russian]{babel}%используем русский и английский языки с переносами
\usepackage{vmargin}
\setpapersize{A4}
\setmarginsrb{26mm}{10mm}{8mm}{8mm}{0pt}{0mm}{0mm}{17mm}
\usepackage{indentfirst}
\sloppy
\usepackage{graphicx} %хотим вставлять в диплом рисунки?
\usepackage{amssymb,amsfonts,amsmath,mathtext,cite,enumerate,float} %подключаем нужные пакеты расширений
\usepackage{titlesec} %хотим вставлять в диплом рисунки?

\begin{document}

\begin{figure}
\quad В какой-то момент при работе за виртуальной машиной у меня высветилось сообщение о том, что память заканчивается. 
\end{figure}

\begin{figure}
\quad Для начала я ввёл в поисковике Linux-а «Диск» и, нажав на иконку диска, убедился насколько он заполнен и сколько памяти осталось. 
\end{figure}

\begin{figure}
\quad Я выключил виртуальную машину, а не сохранил, чтобы можно было изменить её настройки. Далее в меню VirtualBox я выбрал пункт «Файл» и нажал на «Менеджер виртуальных носителей…».
\end{figure}

\begin{figure}
\quad В открывшемся окне я зашел в свойства диска и нашёл его размер. Двигая за ползунок я изменил его объём с 16 гигабайт до 30 и нажал «Применить». 
\end{figure}

\begin{figure}
\quad Когда я снова открыл виртуальную машину и зашёл в «Диск», то увидел, что размер никак не изменился. Я понял, что чтобы увеличить объём памяти жёсткого диска, нужно сделать клон текущего состояния виртуальной машины или удалить снимки. К сожалению изменять размер диска, просто нельзя, пока существуют его точки сохранения. Поэтому я снова выключил машину, и сделал клон её текущего состояния. Он делается точно так же, как когда я делал клон для переноса его с одного устройства на другое. Единственное отличие в том, что я выбрал клонировать не всё, а только текущее состояние. 
\end{figure}

\begin{figure}
\quad Увеличив в «Менеджере виртуальных носителей…» уже объём памяти клона, я также зашёл в настройки виртуальной машины, увеличил размер памяти в «Системе» и выбрал соответствующий диск в «Носителях», так же как я это делал, после переноса клона на другое устройство. Далее я запустил новую машину и зашёл в «Диск», где увидел, что его размер стал больше, и появилась дополнительная область памяти.
\end{figure}

\begin{figure}
\quad Чтобы задействовать его, я нажал на значок настроек, сдвинул ползунок полностью вправо, чтобы можно было использовать всю новую память и нажал «Применить», после чего ввёл свой пароль. Но опять ничего не получилось. Чтобы наконец всё заработало, нужен установочный файл Ubuntu нашей виртуальной машины. Он остался на прошлом устройстве, с которого мы и клонировали машину. Я воспользовался гугл диском, чтобы перенести файл, так как флэшка не позволяет загрузить в неё что-либо весом более 4 гигабайт без её форматирования.
\end{figure}

\end{document}
