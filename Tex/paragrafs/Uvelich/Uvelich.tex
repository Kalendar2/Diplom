\begin{figure}
\quad В какой-то момент при работе за виртуальной машиной, высветилось сообщение о том, что память заканчивается. 
\end{figure}

\begin{figure}
\quad Для начала вводим в поисковике Linux-а «Диск» и, нажав на иконку диска, смотрим насколько он заполнен, и сколько памяти осталось. 
\end{figure}

\begin{figure}
\quad Необходимо выключить виртуальную машину, а не сохранить, чтобы можно было изменить её настройки. Далее в меню VirtualBox выбираем пункт «Файл» и нажимаем на «Менеджер виртуальных носителей…».
\end{figure}

\begin{figure}
\quad В открывшемся окне заходим в свойства диска и находим его размер. Двигая за ползунок, изменяем его объём с 16 гигабайт до 30 и нажимаем «Применить». 
\end{figure}

\begin{figure}
\quad При повторном открытии виртуальной машины было обнаружено, что размер памяти не изменился. Мы пришли к выводу, что для того чтобы увеличить объём памяти жёсткого диска, нужно сделать клон текущего состояния виртуальной машины или удалить снимки. К сожалению изменять размер диска, просто нельзя, пока существуют его точки сохранения. Поэтому снова выключаем машину, и делаем клон её текущего состояния. Он делается точно так же, как когда мы делали клон для переноса его с одного устройства на другое. Единственное отличие в том, что в этот раз выбираем клонировать не всё, а только текущее состояние. 
\end{figure}

\begin{figure}
\quad Увеличив в «Менеджере виртуальных носителей…» уже объём памяти клона, заходим в настройки виртуальной машины, увеличиваем размер памяти в «Системе» и выбраем соответствующий диск в «Носителях», так же как это делали, после переноса клона на другое устройство. Далее запускаем новую машину и заходим в «Диск», где виидим, что его размер стал больше, и появилась дополнительная область памяти.
\end{figure}

\begin{figure}
\quad Чтобы задействовать его, нажиимае на иконку настроек, сдвигаем ползунок до конца вправо, чтобы можно было использовать всю новую память и нажиимаем «Применить», после чего вводим свой пароль. Но опять ничего не получилось. Чтобы наконец всё заработало, нужен установочный файл Ubuntu нашей виртуальной машины. Он остался на прошлом устройстве, с которого мы и клонировали машину. Воспользовавшись гугл диском, перенесли файл, так как флэшка не позволяет загрузить в неё что-либо весом более 4 гигабайт без её форматирования.
\end{figure}