\addcontentsline{toc}{chapter}{Литература}
\begin{thebibliography}{99}

\bibitem{VirtualBox}VirtualBox. Виртуальные диски. Их типы. Расширение виртуального носителя // Хекслет Рус : [сайт]. - 2021. - URL: https://ru.hexlet.io/blog/posts/virtualbox-virtualnye-diski-ih-tipy-rasshirenie-virtualnogo-nositelya (Дата обращения 22.03.2024)

\end{thebibliography}


\\ 2. \quad Колесников Е. Н. Что такое GitHub и как им пользоваться / Е. Н. Колесников // SkillBox Media : [сайт]. - 2024. - URL:https://skillbox.ru/media/code/chto-takoe-github-i-kak-im-polzovatsya/ (Дата обращения: 02.04.2024)
\\3.\quad \textbf{Львовский С. М.} «Набор и верстка в системе LaTeX» - URL: https://old.mccme.ru//free-books//llang/newllang.pdf (дата обращения: 16.04.2024).
\\4.\quad Образовательный портал «РЕШУ ЕГЭ» - URL: https://ege.sdamgia.ru 
\\5.\quad Открытый банк задач ЕГЭ по Математике. Профильный уровень. – URL: https://prof.mathege.ru
\\6.\quad Полноценная платформа для разработчиков для создания, масштабирования и доставки защищенного программного обеспечения GitHub — URL: https://github.com/nickkolok/chas-ege/ (дата обращения: 20.04.2024).  
\\7.\quad Полный интерактивный тест - Тренажёр «Час ЕГЭ» - URL:https://math.vsu.ru/chas-ege/sh/polnmat.html (дата обращения: 18.04.2024).