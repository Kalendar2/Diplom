\quad ЕГЭ (Единый государственный экзамен) – это государственный экзамен, проводимый в России для выпускников школ. ЕГЭ проводится по 14 предметам, включая французский, испанский, немецкий и с 2019 года китайский язык. Обязательными являются русский язык и базовый или профильный модуль по математике.

\quad ЕГЭ проходит каждый год в период с мая по июль. Цель экзамена – определить уровень знаний и умений выпускников и принять решение о поступлении в высшие учебные заведения. Результаты ЕГЭ засчитываются как вступительное испытание в вузы России. Сдают ЕГЭ выпускники средних школ, которые получили документ об окончании общеобразовательной школы. В целом, это молодые люди в возрасте 16-18 лет.

\quad Для успешной сдачи ЕГЭ необходима тщательная подготовка, и немаловажную роль в этом играет сайт «Решу ЕГЭ»[1]. Он может быть полезен в подготовке к экзамену по математике.
Однако за время подготовки к экзаменам школьники быстро сталкиваются с дефицитом заданий, а в некоторых случаях прибегают к списыванию. Проект «Час ЕГЭ» позволяет разрешить эти проблемы. 

\quad «Час ЕГЭ» — компьютерный образовательный проект, разрабатываемый при математическом факультете ВГУ в рамках «OpenSource кластера» и предназначенный для помощи учащимся старших классов подготовиться к тестовой части единого государственного экзамена.

\quad Задания в «Час ЕГЭ» генерируются случайным образом по специализированным алгоритмам, называемых шаблонами, каждый из которых охватывает множество вариантов соответствующей ему задачи. В настоящее время в проекте полностью реализованы тесты по математике с кратким ответом. Планируется с течением времени включить в проект тесты по другим предметам школьной программы. 

\quad Цель - разработка шаблонов, основанных на задачах из ЕГЭ. Они важны, так как за время подготовки к экзаменам, школьники быстро сталкиваются с дефицитом заданий, а в некоторых случаях прибегают к списыванию. Большое и разнообразное количество заданий, генерируемых алгоритмами шаблонов, позволяет объективно оценить знания школьников.
