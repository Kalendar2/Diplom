За всё время работы было разработано следующее количество шаблонов:
\\Тип №10 по арифметической прогрессии: \textbf{9}
\\Тип №10 по геометрической прогрессии: \textbf{2}
\\Тип №6 ЕГЭ профильного уровня: \textbf{3}
\\Тип №8 ЕГЭ профильного уровня: \textbf{1}
\\Тип №12 ЕГЭ профильного уровня: \textbf{12}
\\Тип №13 ЕГЭ профильного уровня: \textbf{1}
\\Тип №16 ЕГЭ базового уровня: \textbf{23}

Также были подготовлены тезисы, которые были отправлены для заочного участия в 74 МСНТК.

Данная работа помогла оказать вклад в развитии образовательного проекта «Час ЕГЭ», а разработанные шаблоны, будут способствовать в подготовке школьников к тестовой части ЕГЭ по математике. 

Помимо этого, сама дипломная работа будет служить руководством для будущих студентов, которые будут также принимать участие в разработке шаблонов и развитии проекта "Час ЕГЭ".


